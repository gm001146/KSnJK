%!TEX root = ../KSnJK.tex


%%%%%%%%%%%%%%%%%%%%%%%%%%%%%%%%%%%%%%%%%%%%%%%%%%%%%%%%%
%% Part basics %%%%%%%%%%%%%%%%%%%%%%%%%%%%%%%%%%%%%%%%%%
%%%%%%%%%%%%%%%%%%%%%%%%%%%%%%%%%%%%%%%%%%%%%%%%%%%%%%%%%
\addtocontents{toc}{\protect\begin{tocBox}{\tmppartnum}}%
\tPart[]{概要}

%%%%%%%%%%%%%%%%%%%%%%%%%%%%%%%%%%%%%%%%%%%%%%%%%%%%%%%%%%
%% chapters %%%%%%%%%%%%%%%%%%%%%%%%%%%%%%%%%%%%%%%%%%%%%%
%%%%%%%%%%%%%%%%%%%%%%%%%%%%%%%%%%%%%%%%%%%%%%%%%%%%%%%%%%
\modHeadchapter[]{作品の概要}



%%%%%%%%%%%%%%%%%%%%%%%%%%%%%%%%%%%%%%%%%%%%%%%%%%%%%%%%%%
%% section 01.01 %%%%%%%%%%%%%%%%%%%%%%%%%%%%%%%%%%%%%%%%%
%%%%%%%%%%%%%%%%%%%%%%%%%%%%%%%%%%%%%%%%%%%%%%%%%%%%%%%%%%
\modHeadsection{制作・販売元}
2012/08/11にコミックマーケット82にて販売。
\begin{enumerate}[label={\sarrow}]
\item 制作・発売元は\href{http://tasofro.net/}{黄昏フロンティア}
\item \href{http://tasofro.net/satori/}{ウェブサイト}は2012/07/22に公開
\item 2024年11月現在、\href{https://www.melonbooks.co.jp/detail/detail.php?product_id=967172}{メロンブックス}・\href{https://www.dlsite.com/home/work/=/product_id/RJ144848.html}{DLsite}にて定価1540円でダウンロード購入可
\end{enumerate}



%\clearpage
%%%%%%%%%%%%%%%%%%%%%%%%%%%%%%%%%%%%%%%%%%%%%%%%%%%%%%%%%%
%% section 01.02 %%%%%%%%%%%%%%%%%%%%%%%%%%%%%%%%%%%%%%%%%
%%%%%%%%%%%%%%%%%%%%%%%%%%%%%%%%%%%%%%%%%%%%%%%%%%%%%%%%%%
\modHeadsection{ストーリー}
ストーリーはオープニングムービーのとおりで、公式の文言の例には以下のものがある。
\begin{enumerate}[label={\sarrow}]
\item 「さとりおねえちゃんがこいしちゃんやペット達を正しい道に導いてあげるアクションパズルゲームです。」
\item 「さとりおねえちゃんを操作して、こいしちゃんを正しい道に導いてあげよう!」
\item 「『矢印板』や『ケーキ』を始めとするユニークアイテムを駆使して、こいしちゃんの旅路を助けよう☆」
\item 「さとりを操作してこいしとペット達を正しい妖怪の道へ引き戻せ!」
\item 「肝心の一人と二匹はさとりの事など何処吹く風のフリーダム 心を鬼にしてビシビシ教育指導!」
\item 「待ち受ける課題は100問以上 様々な仕掛けがこいし達を教育する」
\item 「はたして彼女の教育の末路は?迫り来る信仰の魔の手を打ち倒せるか!?」
\end{enumerate}
また、各ステージのタイトルや背景でもストーリーの進み具合がわかるようになっている。


\clearpage
%%%%%%%%%%%%%%%%%%%%%%%%%%%%%%%%%%%%%%%%%%%%%%%%%%%%%%%%%%
%% section 01.03 %%%%%%%%%%%%%%%%%%%%%%%%%%%%%%%%%%%%%%%%%
%%%%%%%%%%%%%%%%%%%%%%%%%%%%%%%%%%%%%%%%%%%%%%%%%%%%%%%%%%
\modHeadsection{内容の基本事項}
\begin{enumerate}[label={\sarrow}]
\item ステージ(時限)は全部で120ステージ(100ステージ+20ステージ)
\item 100時限目をクリアするとエンディング
\item エンディング後、101-120時限目が追加される
\item 1時限目からスタートし、1ステージクリアすればその次の時限目が選択可能になる
  \begin{enumerate}[label={\sarrow}]
  \item ただし計10ステージをクリアすると、そのときだけ次とその次の時限目も選択可能になる\\
        これは計10ステージクリアするごとにくりかえされる
  \item そのため最短91ステージをクリアすればエンディングに到達可能\\
        (最短90ステージをクリアすれば100時限目に到達可能)
  \item 101-120時限目については、エンディング後同時に全て選択可能となる
  \end{enumerate}
\item こいしは全てのステージで登場し、最初はこいしのみが登場
  \begin{enumerate}[label={\sarrow}]
  \item 21時限目以降からおりん(火焔猫燐)も登場(常に登場するとは限らない)
  \item 41時限目以降からおくう(霊烏路空)も登場(常に登場するとは限らない)
  \item 登場パターンは「こいし」「こいし・おりん」「こいし・おりん・おくう」の3パターンのみ
  \end{enumerate}
\item こいし達が全員プレゼントに正面から触れると、そのステージのクリアとなる
\end{enumerate}



%\clearpage
%%%%%%%%%%%%%%%%%%%%%%%%%%%%%%%%%%%%%%%%%%%%%%%%%%%%%%%%%%
%% section 01.04 %%%%%%%%%%%%%%%%%%%%%%%%%%%%%%%%%%%%%%%%%
%%%%%%%%%%%%%%%%%%%%%%%%%%%%%%%%%%%%%%%%%%%%%%%%%%%%%%%%%%
\modHeadsection{指導の評価}


%%%%%%%%%%%%%%%%%%%%%%%%%%%%%%%%%%%%%%%%%%%%%%%%%%%%%%%%%%
%% subsection 01.04.01 %%%%%%%%%%%%%%%%%%%%%%%%%%%%%%%%%%%
%%%%%%%%%%%%%%%%%%%%%%%%%%%%%%%%%%%%%%%%%%%%%%%%%%%%%%%%%%
\subsection{評価の種類}
各々のステージの結果により、以下の4つの評価がなされる。
\begin{enumerate}[label={\sarrow}]
\item 優:そのステージに定められた時間%
%%% footnote %%%
\footnote{具体的な時間は知らない。}%
%%%%%%%%%%%%%%%%
内にクリアした場合
\item 良:そのステージに定められた時間を超えてクリアした場合
\item 可:「こいしをつよくする」を用いてクリアした場合
\item 指導失敗(評価なし):クリアできなかった場合
\end{enumerate}
途中でやり直すことも可能。


%%%%%%%%%%%%%%%%%%%%%%%%%%%%%%%%%%%%%%%%%%%%%%%%%%%%%%%%%%
%% subsection 01.04.02 %%%%%%%%%%%%%%%%%%%%%%%%%%%%%%%%%%%
%%%%%%%%%%%%%%%%%%%%%%%%%%%%%%%%%%%%%%%%%%%%%%%%%%%%%%%%%%
\subsection{指導失敗の例}
\begin{enumerate}[label={\sarrow}]
\item こいし達のいずれかの「ライフ」が0になり消滅した場合
\item こいし達のいずれかがあな(水・溶岩含む)に落ちた場合
\end{enumerate}
つまり、こいし達のいずれかが画面から消滅した場合といえる。
なお、プレゼントが消滅しても指導失敗とはならないが、事実上の指導失敗である。
%%%%%%%%%%%%%%%%%%%%%%%%%%%%%%%%%%%%%%%%%%%%%
%% marker %%%%%%%%%%%%%%%%%%%%%%%%%%%%%%%%%%%
%%%%%%%%%%%%%%%%%%%%%%%%%%%%%%%%%%%%%%%%%%%%%
\begin{marker}
以降、「こいし達」「さとり達」を以下の意味で用いるものとする。
\begin{enumerate}[label={\sarrow}]
\item こいし達:こいし・おりん・おくう
\item さとり達:さとり・こいし達
\end{enumerate}
\end{marker}
%%%%%%%%%%%%%%%%%%%%%%%%%%%%%%%%%%%%%%%%%%%%%
%%%%%%%%%%%%%%%%%%%%%%%%%%%%%%%%%%%%%%%%%%%%%
%%%%%%%%%%%%%%%%%%%%%%%%%%%%%%%%%%%%%%%%%%%%%


\clearpage
%%%%%%%%%%%%%%%%%%%%%%%%%%%%%%%%%%%%%%%%%%%%%%%%%%%%%%%%%%
%% section 01.05 %%%%%%%%%%%%%%%%%%%%%%%%%%%%%%%%%%%%%%%%%
%%%%%%%%%%%%%%%%%%%%%%%%%%%%%%%%%%%%%%%%%%%%%%%%%%%%%%%%%%
\modHeadsection{アイテム}
さとりはアイテムを持ち上げて、別の場所に置くことができる。

なお、アイテムが初めて登場するステージでは、(クリアするまでは)それらを使用したデモムービーが流れる。


%%%%%%%%%%%%%%%%%%%%%%%%%%%%%%%%%%%%%%%%%%%%%%%%%%%%%%%%%%
%% subsection 01.05.01 %%%%%%%%%%%%%%%%%%%%%%%%%%%%%%%%%%%
%%%%%%%%%%%%%%%%%%%%%%%%%%%%%%%%%%%%%%%%%%%%%%%%%%%%%%%%%%
\subsection{やじるし}
\begin{enumerate}[label={\sarrow}]
\item 1時限目「はじめてのやじるし」で初登場
\item こいし達がやじるしを見つけると、やじるしの向いた方向に進む
\item やじるしがこいし達の進む方向と同じだと、そのままやじるしを壊して進む。
\item こいし達を一定時間停止させる、という意図でも用いることがある
\end{enumerate}


%%%%%%%%%%%%%%%%%%%%%%%%%%%%%%%%%%%%%%%%%%%%%%%%%%%%%%%%%%
%% subsection 01.05.02 %%%%%%%%%%%%%%%%%%%%%%%%%%%%%%%%%%%
%%%%%%%%%%%%%%%%%%%%%%%%%%%%%%%%%%%%%%%%%%%%%%%%%%%%%%%%%%
\subsection{ジャンプだい}
\begin{enumerate}[label={\sarrow}]
\item 8時限目「はじめてのジャンプ」で初登場
\item こいし達がジャンプだいを見つけると1マス先にジャンプする
\item さとりは1段上からジャンプだいに飛び降りるとジャンプすることができる
\item さとりがジャンプだいを持った状態で1段上からこいし達が乗ると、さとりは踏まれずにそのままジャンプする
\end{enumerate}


%%%%%%%%%%%%%%%%%%%%%%%%%%%%%%%%%%%%%%%%%%%%%%%%%%%%%%%%%%
%% subsection 01.05.15 %%%%%%%%%%%%%%%%%%%%%%%%%%%%%%%%%%%
%%%%%%%%%%%%%%%%%%%%%%%%%%%%%%%%%%%%%%%%%%%%%%%%%%%%%%%%%%
\subsection{ばくだん}
\begin{enumerate}[label={\sarrow}]
\item 81時限目「プレゼントをまもれ」で初登場
\item ばくだんがダメージを受けると、爆発する
\item 1段上から落としても、爆発する
\item 爆発すると、ばくだんは消滅し、そのマスと上下左右の4マスに爆風が広がる
\item こいし達が爆風を受けると、ダメージを受ける
\item さとりが爆風を受けると、さとりが消滅し、操作不能になる
\item アイテム・プレゼント・いわ・てきが爆風を受けると、消滅する
\item こいし達がばくだんを見つけると正面に吹き飛ばし、何かに当たると爆発する
\end{enumerate}


%%%%%%%%%%%%%%%%%%%%%%%%%%%%%%%%%%%%%%%%%%%%%%%%%%%%%%%%%%
%% subsection 01.05.03 %%%%%%%%%%%%%%%%%%%%%%%%%%%%%%%%%%%
%%%%%%%%%%%%%%%%%%%%%%%%%%%%%%%%%%%%%%%%%%%%%%%%%%%%%%%%%%
\subsection{ケーキ}
\begin{enumerate}[label={\sarrow}]
\item 3時限目「がめんはじはかべ」で初登場
\item こいし達がケーキ見つけると食べて元気になる(ライフが全回復する)
\item こいし達を一定時間停止させる、という意図でも用いることがある
\end{enumerate}


\clearpage
%%%%%%%%%%%%%%%%%%%%%%%%%%%%%%%%%%%%%%%%%%%%%%%%%%%%%%%%%%
%% subsection 01.05.13 %%%%%%%%%%%%%%%%%%%%%%%%%%%%%%%%%%%
%%%%%%%%%%%%%%%%%%%%%%%%%%%%%%%%%%%%%%%%%%%%%%%%%%%%%%%%%%
\subsection{キノコ}
\begin{enumerate}[label={\sarrow}]
\item 65時限目「おいしいキノコ」で初登場
\item こいし達がキノコを見つけると食べてお元気になる(ライフが全回復し、頭にキノコが生える)
\item やじるしを見て逆の方向に進む
\item ばくだんをその場で爆発させる
\item ダメージを受けると正気に戻る
\item キノコが生えた状態でキノコを食べても変化はない
\item こいし達を一定時間停止させる、という意図でも用いることがある
\end{enumerate}


%%%%%%%%%%%%%%%%%%%%%%%%%%%%%%%%%%%%%%%%%%%%%%%%%%%%%%%%%%
%% subsection 01.05.16 %%%%%%%%%%%%%%%%%%%%%%%%%%%%%%%%%%%
%%%%%%%%%%%%%%%%%%%%%%%%%%%%%%%%%%%%%%%%%%%%%%%%%%%%%%%%%%
\subsection{とうがらし}
\begin{enumerate}[label={\sarrow}]
\item 84時限目「からいたべもの」で初登場
\item こいし達がとうがらしを見つけると食べて元気になる(ライフが全回復する)
\item かべ・いわ・ばくだんにぶつかるまで走り続ける
\item ダメージを受けると正気に戻る
\item 走っている最中に触れたアイテム・プレゼント・いわ・てきは消滅する
\end{enumerate}


%%%%%%%%%%%%%%%%%%%%%%%%%%%%%%%%%%%%%%%%%%%%%%%%%%%%%%%%%%
%% subsection 01.05.17 %%%%%%%%%%%%%%%%%%%%%%%%%%%%%%%%%%%
%%%%%%%%%%%%%%%%%%%%%%%%%%%%%%%%%%%%%%%%%%%%%%%%%%%%%%%%%%
\subsection{あかいばくだん}
\begin{enumerate}[label={\sarrow}]
\item 89時限目「あかいばくだん」で初登場
\item 爆発すると、消滅し、そのマスと上下左右の直線上に爆風が広がる
\item それ以外はばくだんと同じ
\item 直線上のマスになにかがあると、それ以上は爆風は広がらない
\end{enumerate}


\clearpage
%%%%%%%%%%%%%%%%%%%%%%%%%%%%%%%%%%%%%%%%%%%%%%%%%%%%%%%%%%
%% section 01.06 %%%%%%%%%%%%%%%%%%%%%%%%%%%%%%%%%%%%%%%%%
%%%%%%%%%%%%%%%%%%%%%%%%%%%%%%%%%%%%%%%%%%%%%%%%%%%%%%%%%%
\modHeadsection{うごかせないものとゆか}



%%%%%%%%%%%%%%%%%%%%%%%%%%%%%%%%%%%%%%%%%%%%%%%%%%%%%%%%%%
%% subsection 01.06.01 %%%%%%%%%%%%%%%%%%%%%%%%%%%%%%%%%%%
%%%%%%%%%%%%%%%%%%%%%%%%%%%%%%%%%%%%%%%%%%%%%%%%%%%%%%%%%%
\subsection{プレゼント}
\begin{enumerate}[label={\sarrow}]
\item すべてのステージで登場
\item こいし達がプレゼントに正面から触れるとステージクリアになり、お菓子を放出する
\item こいしをつよくしてクリアすると、おかしはほとんど出ない
\item プレゼントが消滅すると、事実上の指導失敗となる
\end{enumerate}


%%%%%%%%%%%%%%%%%%%%%%%%%%%%%%%%%%%%%%%%%%%%%%%%%%%%%%%%%%
%% subsection 01.06.02 %%%%%%%%%%%%%%%%%%%%%%%%%%%%%%%%%%%
%%%%%%%%%%%%%%%%%%%%%%%%%%%%%%%%%%%%%%%%%%%%%%%%%%%%%%%%%%
\subsection{いわ}
\begin{enumerate}[label={\sarrow}]
\item 26時限目「けだまをよけよう」で初登場
\item 上を歩くことができる
\item とうがらしを食べて体当たりすると壊すことができる
\item ばくだん・あかいばくだんの爆風で壊すことができる
\item 同じ段ではかべと同様
\end{enumerate}


%%%%%%%%%%%%%%%%%%%%%%%%%%%%%%%%%%%%%%%%%%%%%%%%%%%%%%%%%%
%% subsection 01.06.03 %%%%%%%%%%%%%%%%%%%%%%%%%%%%%%%%%%%
%%%%%%%%%%%%%%%%%%%%%%%%%%%%%%%%%%%%%%%%%%%%%%%%%%%%%%%%%%
\subsection{スイッチ}
\begin{enumerate}[label={\sarrow}]
\item 53時限目「はじめてのスイッチ」で初登場、デモムービーあり
\item スイッチを踏んだり、上にアイテムを置くと、同色の床が一段高くなる
\item 赤・青・黄・緑の4種類がある
\end{enumerate}


%%%%%%%%%%%%%%%%%%%%%%%%%%%%%%%%%%%%%%%%%%%%%%%%%%%%%%%%%%
%% subsection 01.06.04 %%%%%%%%%%%%%%%%%%%%%%%%%%%%%%%%%%%
%%%%%%%%%%%%%%%%%%%%%%%%%%%%%%%%%%%%%%%%%%%%%%%%%%%%%%%%%%
\subsection{あな}
\begin{enumerate}[label={\sarrow}]
\item 5時限目「あなはおちるよ」で初登場、デモムービーあり
\item さとり達がこのマスに乗ると、落ちる
\item こいし達のいずれかが落ちると、その時点で指導失敗となる
\item さとりが落ちると、画面から消滅し、操作不能となる
\item みず・ようがんのマスもあなと同様
\end{enumerate}


%%%%%%%%%%%%%%%%%%%%%%%%%%%%%%%%%%%%%%%%%%%%%%%%%%%%%%%%%%
%% subsection 01.06.05 %%%%%%%%%%%%%%%%%%%%%%%%%%%%%%%%%%%
%%%%%%%%%%%%%%%%%%%%%%%%%%%%%%%%%%%%%%%%%%%%%%%%%%%%%%%%%%
\subsection{おちるゆか}
\begin{enumerate}[label={\sarrow}]
\item 31時限目「とおるとおちるゆか」で初登場、デモムービーあり
\item さとり達がおちるゆかに乗って離れると崩れ落ちる
\item 崩れ落ちるまで少しだけラグがあり、それまでは通ることができる
\end{enumerate}


%%%%%%%%%%%%%%%%%%%%%%%%%%%%%%%%%%%%%%%%%%%%%%%%%%%%%%%%%%
%% subsection 01.06.06 %%%%%%%%%%%%%%%%%%%%%%%%%%%%%%%%%%%
%%%%%%%%%%%%%%%%%%%%%%%%%%%%%%%%%%%%%%%%%%%%%%%%%%%%%%%%%%
\subsection{すべるゆか}
\begin{enumerate}[label={\sarrow}]
\item 45時限目「すっごいすべるよ」で初登場、デモムービーあり
\item さとり達が通ると、その先のマスになにもない限り進行方向に強制的に進む
\end{enumerate}


%%%%%%%%%%%%%%%%%%%%%%%%%%%%%%%%%%%%%%%%%%%%%%%%%%%%%%%%%%
%% subsection 01.06.07 %%%%%%%%%%%%%%%%%%%%%%%%%%%%%%%%%%%
%%%%%%%%%%%%%%%%%%%%%%%%%%%%%%%%%%%%%%%%%%%%%%%%%%%%%%%%%%
\subsection{おちるすべるゆか}
\begin{enumerate}[label={\sarrow}]
\item 50時限目「すべっておちる」で初登場
\item 「おちるゆか」と「すべるゆか」の両方の性質を持つ
\end{enumerate}


\clearpage
%%%%%%%%%%%%%%%%%%%%%%%%%%%%%%%%%%%%%%%%%%%%%%%%%%%%%%%%%%
%% subsection 01.06.08 %%%%%%%%%%%%%%%%%%%%%%%%%%%%%%%%%%%
%%%%%%%%%%%%%%%%%%%%%%%%%%%%%%%%%%%%%%%%%%%%%%%%%%%%%%%%%%
\subsection{ワープゆか}
\begin{enumerate}[label={\sarrow}]
\item 69時限目「ワープするゆか」で初登場
\item ワープゆかに乗ると、同色のワープゆかへワープする
\item アイテムだけを置いてもワープする
\item ワープゆかの上に何かが乗っていると、ワープできない
\item 赤・青・黄・緑の4種類がある
\end{enumerate}


\clearpage
%%%%%%%%%%%%%%%%%%%%%%%%%%%%%%%%%%%%%%%%%%%%%%%%%%%%%%%%%%
%% section 01.07 %%%%%%%%%%%%%%%%%%%%%%%%%%%%%%%%%%%%%%%%%
%%%%%%%%%%%%%%%%%%%%%%%%%%%%%%%%%%%%%%%%%%%%%%%%%%%%%%%%%%
\modHeadsection{てき}
\begin{enumerate}[label={\sarrow}]
\item こいし達がてきと接触すると、てきは消滅し、こいし達はダメージを受ける
\item さとりはてきと接触すると踏まれる
\item てきは浮いているので、ゆかの影響を受けない
\end{enumerate}
なお、てきが初めて登場するステージでは、(クリアするまでは)それらのデモムービーが流れる。


%%%%%%%%%%%%%%%%%%%%%%%%%%%%%%%%%%%%%%%%%%%%%%%%%%%%%%%%%%
%% subsection 01.07.01 %%%%%%%%%%%%%%%%%%%%%%%%%%%%%%%%%%%
%%%%%%%%%%%%%%%%%%%%%%%%%%%%%%%%%%%%%%%%%%%%%%%%%%%%%%%%%%
\subsection{けだま}
\begin{enumerate}[label={\sarrow}]
\item 25時限目「けだまをたおそう」で初登場
\item 向いている方向に往復移動をし続ける
\item かべ・あな・アイテム・けだまに触れると反対方向に進む
\end{enumerate}


%%%%%%%%%%%%%%%%%%%%%%%%%%%%%%%%%%%%%%%%%%%%%%%%%%%%%%%%%%
%% subsection 01.07.02 %%%%%%%%%%%%%%%%%%%%%%%%%%%%%%%%%%%
%%%%%%%%%%%%%%%%%%%%%%%%%%%%%%%%%%%%%%%%%%%%%%%%%%%%%%%%%%
\subsection{はやいようせい}
\begin{enumerate}[label={\sarrow}]
\item 61時限目「もりのようせい」で初登場
\item はやいようせいはなにもなければ定位置に留まっている
\item はやいようせいが向いている方向にさとり達が居ると、高速で向かってくる
\item さとり達がいなくなると、もとの定位置に向かって進む
\item さとりの場合はその場で踏まれ、起き上がるまでの間はもとの定位置に向かって進む
\item はやいようせいはあなのマスは通らない
\end{enumerate}


%%%%%%%%%%%%%%%%%%%%%%%%%%%%%%%%%%%%%%%%%%%%%%%%%%%%%%%%%%
%% subsection 01.07.03 %%%%%%%%%%%%%%%%%%%%%%%%%%%%%%%%%%%
%%%%%%%%%%%%%%%%%%%%%%%%%%%%%%%%%%%%%%%%%%%%%%%%%%%%%%%%%%
\subsection{おそいようせい}
\begin{enumerate}[label={\sarrow}]
\item 63時限目「おそいようせい」で初登場
\item おそいようせいはなにもなければその場に留まっている
\item おそいようせいの直線上にさとり達が居ると、ゆっくり向かってくる
\item 向かっている最中に他の方向にさとり達がいても、そのまま向かい続ける
\item 複数のおそいようせいが同時にぶつかった場合、減るライフは1のみ
\item さとりの場合はそのマスで踏まれ続けられ、おそいようせいが消えない限り操作不能となる
\item さとりを踏んでいる間はそのマスから動かない
\item おそいようせいはあなのマスは通らない
\end{enumerate}


\clearpage
%%%%%%%%%%%%%%%%%%%%%%%%%%%%%%%%%%%%%%%%%%%%%%%%%%%%%%%%%%
%% section 01.08 %%%%%%%%%%%%%%%%%%%%%%%%%%%%%%%%%%%%%%%%%
%%%%%%%%%%%%%%%%%%%%%%%%%%%%%%%%%%%%%%%%%%%%%%%%%%%%%%%%%%
\modHeadsection{その他の登場人物}


%%%%%%%%%%%%%%%%%%%%%%%%%%%%%%%%%%%%%%%%%%%%%%%%%%%%%%%%%%
%% subsection 01.05.13 %%%%%%%%%%%%%%%%%%%%%%%%%%%%%%%%%%%
%%%%%%%%%%%%%%%%%%%%%%%%%%%%%%%%%%%%%%%%%%%%%%%%%%%%%%%%%%
\subsection{ひじり}
\begin{enumerate}[label={\sarrow}]
\item 100時限目「ひじりをたおせ」でのみ登場
\item ひじりはその場で留まって踊り続けている
\item ダメージを3回受けると消滅する
\end{enumerate}


%%%%%%%%%%%%%%%%%%%%%%%%%%%%%%%%%%%%%%%%%%%%%%%%%%%%%%%%%%
%% subsection 01.05.13 %%%%%%%%%%%%%%%%%%%%%%%%%%%%%%%%%%%
%%%%%%%%%%%%%%%%%%%%%%%%%%%%%%%%%%%%%%%%%%%%%%%%%%%%%%%%%%
\subsection{みこ}
\begin{enumerate}[label={\sarrow}]
\item 120時限目「しんのラスボス」でのみ登場
\item ひじりと同様
\end{enumerate}



\clearpage
%%%%%%%%%%%%%%%%%%%%%%%%%%%%%%%%%%%%%%%%%%%%%%%%%%%%%%%%%%
%% section 01.06 %%%%%%%%%%%%%%%%%%%%%%%%%%%%%%%%%%%%%%%%%
%%%%%%%%%%%%%%%%%%%%%%%%%%%%%%%%%%%%%%%%%%%%%%%%%%%%%%%%%%
\modHeadsection[ライフとダメージ]{ライフ%
%%% footnote %%%
\footnote{「ライフ」についての公式の文言はなく、勝手にそう名付けている。}%
%%%%%%%%%%%%%%%%
とダメージ}


%%%%%%%%%%%%%%%%%%%%%%%%%%%%%%%%%%%%%%%%%%%%%%%%%%%%%%%%%%
%% subsection 01.06.01 %%%%%%%%%%%%%%%%%%%%%%%%%%%%%%%%%%%
%%%%%%%%%%%%%%%%%%%%%%%%%%%%%%%%%%%%%%%%%%%%%%%%%%%%%%%%%%
\subsection{こいし達の場合}

%%%%%%%%%%%%%%%%%%%%%%%%%%%%%%%%%%%%%%%%%%%%%%%%%%%%%%%%%%
%% subsubsection 01.06.01.1 %%%%%%%%%%%%%%%%%%%%%%%%%%%%%%
%%%%%%%%%%%%%%%%%%%%%%%%%%%%%%%%%%%%%%%%%%%%%%%%%%%%%%%%%%
\subsubsection{こいし達のライフ}
\begin{enumerate}[label={\sarrow}]
\item 各々のステージが始まった段階で、こいし達のライフは3
\item こいし達のいずれかのライフが0になると、消滅し指導失敗となる
\item ダメージを受けると、ライフが1減る
\item たべものを食べると、ライフが全回復(ライフ3)する
\item ライフが1になると、ふらつく
\item 「こいしをつよくしてやりなおす」を選ぶと、こいし達のライフが減らなくなる
\end{enumerate}

%%%%%%%%%%%%%%%%%%%%%%%%%%%%%%%%%%%%%%%%%%%%%%%%%%%%%%%%%%
%% subsubsection 01.06.01.2 %%%%%%%%%%%%%%%%%%%%%%%%%%%%%%
%%%%%%%%%%%%%%%%%%%%%%%%%%%%%%%%%%%%%%%%%%%%%%%%%%%%%%%%%%
\subsubsection{こいし達の受けるダメージ}
以下のような場合に、こいし達はダメージを受ける
\begin{enumerate}[label={\sarrow}]
\item かべ・てきにぶつかる
\item こいし達同士でぶつかる
\item 爆風を受ける
\item 1段上からアイテムが当たる
\item さとりに踏まれる
\end{enumerate}
\begin{marker}
プレゼントに触れている間はダメージを受けない
\end{marker}


%%%%%%%%%%%%%%%%%%%%%%%%%%%%%%%%%%%%%%%%%%%%%%%%%%%%%%%%%%
%% subsection 01.06.04 %%%%%%%%%%%%%%%%%%%%%%%%%%%%%%%%%%%
%%%%%%%%%%%%%%%%%%%%%%%%%%%%%%%%%%%%%%%%%%%%%%%%%%%%%%%%%%
\subsection{てきの場合}

%%%%%%%%%%%%%%%%%%%%%%%%%%%%%%%%%%%%%%%%%%%%%%%%%%%%%%%%%%
%% subsubsection 01.06.04.1 %%%%%%%%%%%%%%%%%%%%%%%%%%%%%%
%%%%%%%%%%%%%%%%%%%%%%%%%%%%%%%%%%%%%%%%%%%%%%%%%%%%%%%%%%
\subsubsection{てきのライフ}
\begin{enumerate}[label={\sarrow}]
\item 各々のステージが始まった段階で、てきのライフは1
\item ライフが0になると、消滅する
\end{enumerate}

%%%%%%%%%%%%%%%%%%%%%%%%%%%%%%%%%%%%%%%%%%%%%%%%%%%%%%%%%%
%% subsubsection 01.06.04.1 %%%%%%%%%%%%%%%%%%%%%%%%%%%%%%
%%%%%%%%%%%%%%%%%%%%%%%%%%%%%%%%%%%%%%%%%%%%%%%%%%%%%%%%%%
\subsubsection{てきの受けるダメージ}
以下のような場合に、てきはダメージを受ける
\begin{enumerate}[label={\sarrow}]
\item こいし達にぶつかる
\item 1段上からさとり達に踏まれる
\item 1段上からアイテムが当たる
\end{enumerate}


%%%%%%%%%%%%%%%%%%%%%%%%%%%%%%%%%%%%%%%%%%%%%%%%%%%%%%%%%%
%% subsection 01.06.02 %%%%%%%%%%%%%%%%%%%%%%%%%%%%%%%%%%%
%%%%%%%%%%%%%%%%%%%%%%%%%%%%%%%%%%%%%%%%%%%%%%%%%%%%%%%%%%
\subsection{さとりの場合}
\begin{enumerate}[label={\sarrow}]
\item さとりは基本的にダメージを受けない
\item 爆風を受けた場合、消滅する
\end{enumerate}


\clearpage
%%%%%%%%%%%%%%%%%%%%%%%%%%%%%%%%%%%%%%%%%%%%%%%%%%%%%%%%%%
%% subsection 01.06.03 %%%%%%%%%%%%%%%%%%%%%%%%%%%%%%%%%%%
%%%%%%%%%%%%%%%%%%%%%%%%%%%%%%%%%%%%%%%%%%%%%%%%%%%%%%%%%%
\subsection{ひじり・みこの場合}
\begin{enumerate}[label={\sarrow}]
\item 各々のステージが始まった段階で、ひじり・みこのライフは3
\item 爆風を受けると、ダメージを受け、ライフが1減る
\item ライフが0になると、消滅する
\end{enumerate}



%\clearpage
%%%%%%%%%%%%%%%%%%%%%%%%%%%%%%%%%%%%%%%%%%%%%%%%%%%%%%%%%%
%% section 01.09 %%%%%%%%%%%%%%%%%%%%%%%%%%%%%%%%%%%%%%%%%
%%%%%%%%%%%%%%%%%%%%%%%%%%%%%%%%%%%%%%%%%%%%%%%%%%%%%%%%%%
\modHeadsection{実績}
以下を満たすと、さとり達のアイコンが表示される
\begin{enumerate}[label={\sarrow}]
\item こいし:100時限目をクリアする
\item おりん:120時限目をクリアする
\item おくう:120時限すべてをクリアする
\item さとり:120時限すべてを優評価でクリアする
\end{enumerate}



%%%%%%%%%%%%%%%%%%%%%%%%%%%%%%%%%%%%%%%%%%%%%%%%%%%%%%%%%%
%% section 01.10 %%%%%%%%%%%%%%%%%%%%%%%%%%%%%%%%%%%%%%%%%
%%%%%%%%%%%%%%%%%%%%%%%%%%%%%%%%%%%%%%%%%%%%%%%%%%%%%%%%%%
\modHeadsection{その他}


%%%%%%%%%%%%%%%%%%%%%%%%%%%%%%%%%%%%%%%%%%%%%%%%%%%%%%%%%%
%% subsection 01.10.01 %%%%%%%%%%%%%%%%%%%%%%%%%%%%%%%%%%%
%%%%%%%%%%%%%%%%%%%%%%%%%%%%%%%%%%%%%%%%%%%%%%%%%%%%%%%%%%
\subsection{?\hspace{0pt}マーク}
こいし達が次のマスに移動し始めた直後にアイテム等でブロックすると、?\hspace{0pt}のふきだしを出す。
\begin{enumerate}[label={\sarrow}]
\item ?\hspace{0pt}が出ている間は動きを止める
\item 狙って出すのは難しい(フレーム単位の操作)
\end{enumerate}



%%%%%%%%%%%%%%%%%%%%%%%%%%%%%%%%%%%%%%%%%%%%%%%%%%%%%%%%%%
%% subsection 01.10.01 %%%%%%%%%%%%%%%%%%%%%%%%%%%%%%%%%%%
%%%%%%%%%%%%%%%%%%%%%%%%%%%%%%%%%%%%%%%%%%%%%%%%%%%%%%%%%%
\subsection{さとり達の表情}
盤内のさとり達の状況に応じて、外側のさとり達の表情が変わる。

%%%%%%%%%%%%%%%%%%%%%%%%%%%%%%%%%%%%%%%%%%%%%%%%%%%%%%%%%%
%% subsubsection 01.10.02.1 %%%%%%%%%%%%%%%%%%%%%%%%%%%%%%
%%%%%%%%%%%%%%%%%%%%%%%%%%%%%%%%%%%%%%%%%%%%%%%%%%%%%%%%%%
\subsubsection{さとりの表情}
\begin{enumerate}[label={\sarrow}]
\item 踏まれている間
\item 消滅している間
\item 目標達成したとき
\end{enumerate}

%%%%%%%%%%%%%%%%%%%%%%%%%%%%%%%%%%%%%%%%%%%%%%%%%%%%%%%%%%
%% subsubsection 01.10.02.1 %%%%%%%%%%%%%%%%%%%%%%%%%%%%%%
%%%%%%%%%%%%%%%%%%%%%%%%%%%%%%%%%%%%%%%%%%%%%%%%%%%%%%%%%%
\subsubsection{こいし達の表情}
\begin{enumerate}[label={\sarrow}]
\item ダメージを受けたとき
\item ライフが1のとき
\item 頭にキノコが生えているとき
\item とうがらしを食べて突進しているとき
\item プレゼントに触れたとき
\end{enumerate}


%%%%%%%%%%%%%%%%%%%%%%%%%%%%%%%%%%%%%%%%%%%%%%%%%%%%%%%%%%
%% subsection 01.10.02 %%%%%%%%%%%%%%%%%%%%%%%%%%%%%%%%%%%
%%%%%%%%%%%%%%%%%%%%%%%%%%%%%%%%%%%%%%%%%%%%%%%%%%%%%%%%%%
\subsection{早送り機能}
ゲーム速度を2倍にする早送り機能がある
\begin{enumerate}[label={\sarrow}]
\item 早送りしている間は、操作できない
\item 多くのステージで、優評価を得るために必須の機能
\end{enumerate}




%%%%%%%%%%%%%%%%%%%%%%%%%%%%%%%%%%%%%%%%%%%%%%%%%%%%%%%%%%
%% chapters %%%%%%%%%%%%%%%%%%%%%%%%%%%%%%%%%%%%%%%%%%%%%%
%%%%%%%%%%%%%%%%%%%%%%%%%%%%%%%%%%%%%%%%%%%%%%%%%%%%%%%%%%
\modHeadchapter[]{RTAの概要}



%%%%%%%%%%%%%%%%%%%%%%%%%%%%%%%%%%%%%%%%%%%%%%%%%%%%%%%%%%
%% section 02.01 %%%%%%%%%%%%%%%%%%%%%%%%%%%%%%%%%%%%%%%%%
%%%%%%%%%%%%%%%%%%%%%%%%%%%%%%%%%%%%%%%%%%%%%%%%%%%%%%%%%%
\modHeadsection{カテゴリ}
2024/11現在、\href{https://www.speedrun.com/komeiji_satori_no_jousou_kyouiku}{SRC}には以下のカテゴリがある。


%%%%%%%%%%%%%%%%%%%%%%%%%%%%%%%%%%%%%%%%%%%%%%%%%%%%%%%%%%
%% subsection 02.01.01 %%%%%%%%%%%%%%%%%%%%%%%%%%%%%%%%%%%
%%%%%%%%%%%%%%%%%%%%%%%%%%%%%%%%%%%%%%%%%%%%%%%%%%%%%%%%%%
\subsection{any\%}
\begin{enumerate}[label={\sarrow}]
\item 100時限目をクリアする(エンディングに到達する)
\item 言いかえると、こいしのアイコンを取得する
\end{enumerate}


%%%%%%%%%%%%%%%%%%%%%%%%%%%%%%%%%%%%%%%%%%%%%%%%%%%%%%%%%%
%% subsection 02.01.02 %%%%%%%%%%%%%%%%%%%%%%%%%%%%%%%%%%%
%%%%%%%%%%%%%%%%%%%%%%%%%%%%%%%%%%%%%%%%%%%%%%%%%%%%%%%%%%
\subsection{any\% - No Strong Koishi}
\begin{enumerate}[label={\sarrow}]
\item any\%と基本的に同じ
\item ただし、「こいしをつよくしてやりなおす」を使用しない
\item 言いかえると、どのステージも優または良の評価でクリアする
\end{enumerate}


%%%%%%%%%%%%%%%%%%%%%%%%%%%%%%%%%%%%%%%%%%%%%%%%%%%%%%%%%%
%% subsection 02.01.03 %%%%%%%%%%%%%%%%%%%%%%%%%%%%%%%%%%%
%%%%%%%%%%%%%%%%%%%%%%%%%%%%%%%%%%%%%%%%%%%%%%%%%%%%%%%%%%
\subsection{any\% - Trial}
\begin{enumerate}[label={\sarrow}]
\item 体験版の1-10時限目すべてをクリア
\end{enumerate}
\begin{marker}
体験版は10ステージともすべて本編のものとは異なりオリジナルのものである
\end{marker}



%%%%%%%%%%%%%%%%%%%%%%%%%%%%%%%%%%%%%%%%%%%%%%%%%%%%%%%%%%
%% section 02.02 %%%%%%%%%%%%%%%%%%%%%%%%%%%%%%%%%%%%%%%%%
%%%%%%%%%%%%%%%%%%%%%%%%%%%%%%%%%%%%%%%%%%%%%%%%%%%%%%%%%%
\modHeadsection{SRC以外の記録}
\begin{enumerate}[label={\sarrow}]
\item Syuraさんの\href{https://pastebin.com/0nH3mJSZ}{1-100時限目のPB}
\item 1-120時限目の\href{https://www.nicovideo.jp/watch/sm18710850}{全優評価クリアの動画}(ニコニコ動画)
\end{enumerate}



\clearpage
%%%%%%%%%%%%%%%%%%%%%%%%%%%%%%%%%%%%%%%%%%%%%%%%%%%%%%%%%%
%% section 02.03 %%%%%%%%%%%%%%%%%%%%%%%%%%%%%%%%%%%%%%%%%
%%%%%%%%%%%%%%%%%%%%%%%%%%%%%%%%%%%%%%%%%%%%%%%%%%%%%%%%%%
\modHeadsection{テクニック}
最短手順で進めるのは言うまでもなく、基本的には早送りをどれだけできるかの勝負。
その他には、
\begin{enumerate}[label={\sarrow}]
\item 操作が独特でボタン入力を受け付けないことがあるため、連射機能を使用\\
      慣れるまではミスが頻発するけど、確実にこのほうが早い\\
      連射機能を用いても入力を受け付けないことが稀にある
\item さとりはアイテムをおくよりも1マス移動するほうが早い\\
      最後に踏まれる形でクリアすることが多いのはそのため
\item さとりがアイテムを持ち上げた直後に早送りをすると、0.1秒ほど縮まる(理由は知らない)\\
      アイテムを持ち上げた直後に少しだけ早送りしていることが多いのはそのため\\
      おそらく1-2fくらい早送りになっていればよいかと思われる(体感)
\item あるステージをクリアし、次のステージをいかに早く選べるか\\
      よくある失敗が、カーソルが動かせずもとのステージを選んでしまうミス(秒単位のロス)
\item ステージの選択で右ボタンを押すと、選択できる最後のステージにカーソルを移動できる\\
      個人的には右ボタンで選ぶほうがやりやすい\\
      1-50時限目くらいまで選ぶ順番が前後しているのはそのため
\item さとりが踏まれる直前にアイテムをおくと、アイテムをおくモーションが省略され早くなる\\
      より直前であるほど時間の短縮になる\\
      フレーム単位の操作になるのでRTAではまずしない(個別のPB狙いではする)
\end{enumerate}



\clearpage
~\vfill
\begin{\Columnname}{小話}
\begin{enumerate}[label={\sarrow}]
\item たまたまお店で見かけて存在を知り、衝動買い

\tcbline*
\item なにこのゲームすてきTAS作りたい
\item でもhourglassで動かない
\item ほなRTAか(フレーム単位の操作なんてそんなないやろ)の精神
\end{enumerate}

\tcbline*
\begin{enumerate}[label={\sarrow}]
\item 手順を調べるためにSRCを見ると、ページはあるが記録動画がない(2020年頃)
\item \href{https://www.speedrun.com/users/Syura}{Syuraさん}という方が2015年頃にRTAをしていたらしいが、動画は削除したらしい
\item モデレータが不在で内容も整理されてなく、申請すれば自動的に承認されるような状態に
\end{enumerate}

\tcbline*
\begin{enumerate}[label={\sarrow}]
\item SASさんが\href{https://www.speedrun.com/komeiji_satori_no_jousou_kyouiku/runs/znol46vy}{RTA}を走られる(56:58)
\item それを受け、参考になればとSyuraさんが5年ほど前の\href{https://www.speedrun.com/komeiji_satori_no_jousou_kyouiku/runs/m3n4v8qy}{記録動画}を再投稿(50:21)\\
      さらに、自身の1-100時限目のPBを\href{https://pastebin.com/0nH3mJSZ}{公開}
\item それを受け、SASさんが自身の記録を\href{https://www.speedrun.com/komeiji_satori_no_jousou_kyouiku/runs/m3ro3edm}{更新}(49:57)

\tcbline*
\item これら4つの記録と、ニコニコ動画にあった\href{https://www.nicovideo.jp/watch/sm18710850}{全優評価クリアの記録}をexcelにまとめて比較
\item 8-9ヶ月ほど各ステージを眺め続け、全120ステージ中119ステージを更新
\item 初めての試走の記録がちょうど50分くらいで、WRを更新できることを確信
\item 現在の記録(any\%, 45:08)に至る(45分切りたい)

\tcbline*
\item SRCのページを整理したく、モデレータにしてもらうよう要望するも却下される(2回)
\item じゃあWRの記録を提出するからと言ったら承認された
\item 改めてカテゴリany\%, any\%-NSK, any\%-Trialを創設し、記録を整理
\item ついでにany\%-TrialのWR記録も更新

\tcbline*
\item 2024/11現在any\%-NSKのWRは\href{https://www.twitch.tv/videos/1689044910}{45:23}(ミスが多いのでSRCには投稿していない)
\item any\%, any\%-NSKのWR記録とも、旧手順の箇所がいくつかあるので、更新は十分可能

\tcbline*
\item なおこのゲームの短縮するコツは、短縮したいステージを延べ半日以上くらい眺め続けてたら自ずと解法が見えてくる(おすすめ)
\item フレーム単位の操作がとても要求されるので、そういう意味ではやるんじゃなかったと思ってるRTA
\end{enumerate}
\end{\Columnname}


%%%%%%%%%%%%%%%%%%%%%%%%%%%%%%%%%%%%%%%%%%%%%%%%%%%%%%%%%
%% Appendices %%%%%%%%%%%%%%%%%%%%%%%%%%%%%%%%%%%%%%%%%%%
%%%%%%%%%%%%%%%%%%%%%%%%%%%%%%%%%%%%%%%%%%%%%%%%%%%%%%%%%
\begin{appendices}
%\Appendixpart
\end{appendices}

\addtocontents{toc}{\protect\end{tocBox}}
