%!TEX root = ../KSnJK.tex


%%%%%%%%%%%%%%%%%%%%%%%%%%%%%%%%%%%%%%%%%%%%%%%%%%%%%%%%%
%% Part basics %%%%%%%%%%%%%%%%%%%%%%%%%%%%%%%%%%%%%%%%%%
%%%%%%%%%%%%%%%%%%%%%%%%%%%%%%%%%%%%%%%%%%%%%%%%%%%%%%%%%
\addtocontents{toc}{\protect\begin{tocBox}{\tmppartnum}}%
\tPart[]{概要}

%%%%%%%%%%%%%%%%%%%%%%%%%%%%%%%%%%%%%%%%%%%%%%%%%%%%%%%%%%
%% chapters %%%%%%%%%%%%%%%%%%%%%%%%%%%%%%%%%%%%%%%%%%%%%%
%%%%%%%%%%%%%%%%%%%%%%%%%%%%%%%%%%%%%%%%%%%%%%%%%%%%%%%%%%
\modHeadchapter[]{作品の概要}



%%%%%%%%%%%%%%%%%%%%%%%%%%%%%%%%%%%%%%%%%%%%%%%%%%%%%%%%%%
%% section 01.01 %%%%%%%%%%%%%%%%%%%%%%%%%%%%%%%%%%%%%%%%%
%%%%%%%%%%%%%%%%%%%%%%%%%%%%%%%%%%%%%%%%%%%%%%%%%%%%%%%%%%
\modHeadsection{制作・販売元}
2012/08/11にコミックマーケット82にて販売。
\begin{enumerate}[label={\sarrow}]
\item 制作・発売元は\href{http://tasofro.net/}{黄昏フロンティア}
\item \href{http://tasofro.net/satori/}{ウェブサイト}は2012/07/22に公開
\item 2024年11月現在、\href{https://www.melonbooks.co.jp/detail/detail.php?product_id=967172}{メロンブックス}・\href{https://www.dlsite.com/home/work/=/product_id/RJ144848.html}{DLsite}にて定価1540円でダウンロード購入可
\end{enumerate}



%\clearpage
%%%%%%%%%%%%%%%%%%%%%%%%%%%%%%%%%%%%%%%%%%%%%%%%%%%%%%%%%%
%% section 01.02 %%%%%%%%%%%%%%%%%%%%%%%%%%%%%%%%%%%%%%%%%
%%%%%%%%%%%%%%%%%%%%%%%%%%%%%%%%%%%%%%%%%%%%%%%%%%%%%%%%%%
\modHeadsection{ストーリー}
ストーリーはオープニングムービーのとおりで、公式の文言の例には以下のものがある。
\begin{enumerate}[label={\sarrow}]
\item 「さとりおねえちゃんがこいしちゃんやペット達を正しい道に導いてあげるアクションパズルゲームです。」
\item 「さとりおねえちゃんを操作して、こいしちゃんを正しい道に導いてあげよう!」
\item 「『矢印板』や『ケーキ』を始めとするユニークアイテムを駆使して、こいしちゃんの旅路を助けよう☆」
\item 「さとりを操作してこいしとペット達を正しい妖怪の道へ引き戻せ!」
\item 「肝心の一人と二匹はさとりの事など何処吹く風のフリーダム 心を鬼にしてビシビシ教育指導!」
\item 「待ち受ける課題は100問以上 様々な仕掛けがこいし達を教育する」
\item 「はたして彼女の教育の末路は? 迫り来る信仰の魔の手を打ち倒せるか!?」
\end{enumerate}



\clearpage
%%%%%%%%%%%%%%%%%%%%%%%%%%%%%%%%%%%%%%%%%%%%%%%%%%%%%%%%%%
%% section 01.03 %%%%%%%%%%%%%%%%%%%%%%%%%%%%%%%%%%%%%%%%%
%%%%%%%%%%%%%%%%%%%%%%%%%%%%%%%%%%%%%%%%%%%%%%%%%%%%%%%%%%
\modHeadsection{内容の基本事項}
\begin{enumerate}[label={\sarrow}]
\item ステージ(時限)は全部で120時限(100時限+20時限)
\item 100時限目をクリアするとエンディング
\item エンディング後、101-120時限目が追加される
\item 1時限目からスタートし、1時限クリアすればその次の時限が選択可能になる
  \begin{enumerate}[label={\sarrow}]
  \item ただし計10時限をクリアすると、そのときだけ次とその次の時限も選択可能になる\\
        これは計10時限クリアするごとにくりかえされる
  \item そのため最短91時限をクリアすればエンディングに到達可能\\
        (最短90時限をクリアすれば100時限目に到達可能)
  \item 101-120時限目については、エンディング後同時に全て選択可能となる
  \end{enumerate}
\item こいしは全ての時限目で登場し、最初はこいしのみが登場
  \begin{enumerate}[label={\sarrow}]
  \item 21時限目以降からおりん(火焔猫燐)も登場(常に登場するとは限らない)
  \item 41時限目以降からおくう(霊烏路空)も登場(常に登場するとは限らない)
  \item 登場パターンは「こいし」「こいし・おりん」「こいし・おりん・おくう」の3パターンのみ
  \end{enumerate}
\item こいし達が全員プレゼント箱のとなりのマスに到達すると、その時限目のクリアとなる
\end{enumerate}



%\clearpage
%%%%%%%%%%%%%%%%%%%%%%%%%%%%%%%%%%%%%%%%%%%%%%%%%%%%%%%%%%
%% section 01.04 %%%%%%%%%%%%%%%%%%%%%%%%%%%%%%%%%%%%%%%%%
%%%%%%%%%%%%%%%%%%%%%%%%%%%%%%%%%%%%%%%%%%%%%%%%%%%%%%%%%%
\modHeadsection{指導の評価}


%%%%%%%%%%%%%%%%%%%%%%%%%%%%%%%%%%%%%%%%%%%%%%%%%%%%%%%%%%
%% subsection 01.04.01 %%%%%%%%%%%%%%%%%%%%%%%%%%%%%%%%%%%
%%%%%%%%%%%%%%%%%%%%%%%%%%%%%%%%%%%%%%%%%%%%%%%%%%%%%%%%%%
\subsection{評価の種類}
各々の時限目の結果により、以下の4つの評価がなされる。
\begin{enumerate}[label={\sarrow}]
\item 優:その時限目に定められた時間%
%%% footnote %%%
\footnote{具体的な時間は知らない。}%
%%%%%%%%%%%%%%%%
内にクリアした場合
\item 良:その時限目に定められた時間を超えてクリアした場合
\item 可:「こいしをつよくする」を用いてクリアした場合
\item 指導失敗(評価なし):クリアできなかった場合
\end{enumerate}
途中でやり直すことも可能。


%%%%%%%%%%%%%%%%%%%%%%%%%%%%%%%%%%%%%%%%%%%%%%%%%%%%%%%%%%
%% subsection 01.04.02 %%%%%%%%%%%%%%%%%%%%%%%%%%%%%%%%%%%
%%%%%%%%%%%%%%%%%%%%%%%%%%%%%%%%%%%%%%%%%%%%%%%%%%%%%%%%%%
\subsection{指導失敗の例}
\begin{enumerate}[label={\sarrow}]
\item こいし達のいずれかの「ライフ」が0になり消滅した場合
\item こいし達のいずれかがあな(水・溶岩含む)に落ちた場合
\end{enumerate}
つまり、こいし達のいずれかが画面から消滅した場合といえる。
なお、プレゼントが消滅しても指導失敗とはならないが、事実上の指導失敗である。
%%%%%%%%%%%%%%%%%%%%%%%%%%%%%%%%%%%%%%%%%%%%%
%% marker %%%%%%%%%%%%%%%%%%%%%%%%%%%%%%%%%%%
%%%%%%%%%%%%%%%%%%%%%%%%%%%%%%%%%%%%%%%%%%%%%
\begin{marker}
以降、「こいし達」「さとり達」を以下の意味で用いるものとする。
\begin{enumerate}[label={\sarrow}]
\item こいし達:こいし・おりん・おくう
\item さとり達:さとり・こいし達
\end{enumerate}
\end{marker}
%%%%%%%%%%%%%%%%%%%%%%%%%%%%%%%%%%%%%%%%%%%%%
%%%%%%%%%%%%%%%%%%%%%%%%%%%%%%%%%%%%%%%%%%%%%
%%%%%%%%%%%%%%%%%%%%%%%%%%%%%%%%%%%%%%%%%%%%%


\clearpage
%%%%%%%%%%%%%%%%%%%%%%%%%%%%%%%%%%%%%%%%%%%%%%%%%%%%%%%%%%
%% section 01.05 %%%%%%%%%%%%%%%%%%%%%%%%%%%%%%%%%%%%%%%%%
%%%%%%%%%%%%%%%%%%%%%%%%%%%%%%%%%%%%%%%%%%%%%%%%%%%%%%%%%%
\modHeadsection{アイテムの種類}
こいし達を指導するためのアイテムがいくつか登場する。
それらが初めて登場する時限目では、(クリアするまでは)それらを使用したデモムービーが流れる。

なお、プレゼント以外はすべてさとりが持ち運ぶことができる。


%%%%%%%%%%%%%%%%%%%%%%%%%%%%%%%%%%%%%%%%%%%%%%%%%%%%%%%%%%
%% subsection 01.05.01 %%%%%%%%%%%%%%%%%%%%%%%%%%%%%%%%%%%
%%%%%%%%%%%%%%%%%%%%%%%%%%%%%%%%%%%%%%%%%%%%%%%%%%%%%%%%%%
\subsection{プレゼント}
\begin{enumerate}[label={\sarrow}]
\item すべての時限目で登場
\item プレゼントの上下左右にこいし達が全員接すると、その時限目がクリアとなる
\item 可の評価でクリアすると、おかしはほとんど出ない
\item プレゼントが消滅すると、事実上の指導失敗となる
\end{enumerate}


%%%%%%%%%%%%%%%%%%%%%%%%%%%%%%%%%%%%%%%%%%%%%%%%%%%%%%%%%%
%% subsection 01.05.02 %%%%%%%%%%%%%%%%%%%%%%%%%%%%%%%%%%%
%%%%%%%%%%%%%%%%%%%%%%%%%%%%%%%%%%%%%%%%%%%%%%%%%%%%%%%%%%
\subsection{やじるし}
\begin{enumerate}[label={\sarrow}]
\item 1時限目「はじめてのやじるし」で初登場
\item こいし達が接すると、やじるしの方向に進む
\item やじるしの向きがこいし達の向きと同じ場合、やじるしは破壊され消滅する
\end{enumerate}


%%%%%%%%%%%%%%%%%%%%%%%%%%%%%%%%%%%%%%%%%%%%%%%%%%%%%%%%%%
%% subsection 01.05.03 %%%%%%%%%%%%%%%%%%%%%%%%%%%%%%%%%%%
%%%%%%%%%%%%%%%%%%%%%%%%%%%%%%%%%%%%%%%%%%%%%%%%%%%%%%%%%%
\subsection{ケーキ}
\begin{enumerate}[label={\sarrow}]
\item 3時限目「がめんはじはかべ」で初登場
\item こいし達が接すると、食べる(モーションがかわいい)
\item 食べると、ライフが全回復する
\item こいし達を一定時間停止させる、という意図でも用いられる
\end{enumerate}


%%%%%%%%%%%%%%%%%%%%%%%%%%%%%%%%%%%%%%%%%%%%%%%%%%%%%%%%%%
%% subsection 01.05.05 %%%%%%%%%%%%%%%%%%%%%%%%%%%%%%%%%%%
%%%%%%%%%%%%%%%%%%%%%%%%%%%%%%%%%%%%%%%%%%%%%%%%%%%%%%%%%%
\subsection{ばね}
\begin{enumerate}[label={\sarrow}]
\item 8時限目「はじめてのジャンプ」で初登場
\item こいし達が接すると、進行方向に1マス飛ばす形でジャンプする
\item さとりはひとつうえの段から乗らないとジャンプできない
\item さとりがばねを持った状態でひとつうえの段からこいし達が乗ると、さとりは踏まれずにそのままジャンプする
\end{enumerate}


%%%%%%%%%%%%%%%%%%%%%%%%%%%%%%%%%%%%%%%%%%%%%%%%%%%%%%%%%%
%% subsection 01.05.13 %%%%%%%%%%%%%%%%%%%%%%%%%%%%%%%%%%%
%%%%%%%%%%%%%%%%%%%%%%%%%%%%%%%%%%%%%%%%%%%%%%%%%%%%%%%%%%
\subsection{キノコ}
\begin{enumerate}[label={\sarrow}]
\item 65時限目「おいしいキノコ」で初登場
\item こいし達が接すると、食べて頭にキノコが生える
\item 食べると、ライフが全回復する
\item キノコが生えた状態でやじるしに接すると、やじるし向きと反対方向に進む
\item ダメージを受けると、もとに戻る
\item キノコが生えた状態でキノコを食べても変化はない
\item こいし達を一定時間停止させる、という意図でも用いられる
\end{enumerate}


\clearpage
%%%%%%%%%%%%%%%%%%%%%%%%%%%%%%%%%%%%%%%%%%%%%%%%%%%%%%%%%%
%% subsection 01.05.15 %%%%%%%%%%%%%%%%%%%%%%%%%%%%%%%%%%%
%%%%%%%%%%%%%%%%%%%%%%%%%%%%%%%%%%%%%%%%%%%%%%%%%%%%%%%%%%
\subsection{ばくだん}
\begin{enumerate}[label={\sarrow}]
\item 81時限目「プレゼントをまもれ」で初登場
\item ばくだんがダメージを受けると、爆発する
\item ひとつうえの段から落としても、爆発する
\item 爆発すると、ばくだんは消滅し、そのマスと上下左右の4マスに爆風が広がる
\item こいし達が爆風を受けると、ダメージを受ける
\item さとりが爆風を受けると、さとりが消滅し、操作不能になる
\item アイテム・けだま・ようせい・ゆっくりようせいが爆風を受けると、消滅する
\item こいし達がばくだんに接すると、進行方向にばくだんを突き飛ばす
\item 突き飛ばされたばくだんは、なにかにぶつかるまで突進する
\end{enumerate}


%%%%%%%%%%%%%%%%%%%%%%%%%%%%%%%%%%%%%%%%%%%%%%%%%%%%%%%%%%
%% subsection 01.05.16 %%%%%%%%%%%%%%%%%%%%%%%%%%%%%%%%%%%
%%%%%%%%%%%%%%%%%%%%%%%%%%%%%%%%%%%%%%%%%%%%%%%%%%%%%%%%%%
\subsection{とうがらし}
\begin{enumerate}[label={\sarrow}]
\item 84時限目「からいたべもの」で初登場
\item こいし達が接すると、食べる
\item 食べると、ライフが全回復する
\item 食べると、かべかばくだんに当たるまで進行方向に突進する
\item 突進中に触れたアイテム・けだま・ようせい・ゆっくりようせいは消滅する
\end{enumerate}


%%%%%%%%%%%%%%%%%%%%%%%%%%%%%%%%%%%%%%%%%%%%%%%%%%%%%%%%%%
%% subsection 01.05.17 %%%%%%%%%%%%%%%%%%%%%%%%%%%%%%%%%%%
%%%%%%%%%%%%%%%%%%%%%%%%%%%%%%%%%%%%%%%%%%%%%%%%%%%%%%%%%%
\subsection{あかいばくだん}
\begin{enumerate}[label={\sarrow}]
\item 89時限目「あかいばくだん」で初登場
\item 爆発すると、消滅し、そのマスと上下左右の直線上に爆風が広がる
\item それ以外はばくだんと同じ
\item 直線上のマスになにかがあると、それ以上は爆風は広がらない
\end{enumerate}


\clearpage
%%%%%%%%%%%%%%%%%%%%%%%%%%%%%%%%%%%%%%%%%%%%%%%%%%%%%%%%%%
%% section 01.06 %%%%%%%%%%%%%%%%%%%%%%%%%%%%%%%%%%%%%%%%%
%%%%%%%%%%%%%%%%%%%%%%%%%%%%%%%%%%%%%%%%%%%%%%%%%%%%%%%%%%
\modHeadsection{敵キャラクタの種類}
時限目によっては敵キャラクタがいくつか登場する。
それらが初めて登場する時限目では、(クリアするまでは)それらを使用したデモムービーが流れる。


%%%%%%%%%%%%%%%%%%%%%%%%%%%%%%%%%%%%%%%%%%%%%%%%%%%%%%%%%%
%% subsection 01.06.01 %%%%%%%%%%%%%%%%%%%%%%%%%%%%%%%%%%%
%%%%%%%%%%%%%%%%%%%%%%%%%%%%%%%%%%%%%%%%%%%%%%%%%%%%%%%%%%
\subsection{けだま}
\begin{enumerate}[label={\sarrow}]
\item 25時限目「けだまをたおそう」で初登場
\item こいし達が接すると、ライフが1つ減り、けだまは消滅する
\item さとりが接すると、そのマスでけだまに踏まれ続けられ、けだまが消えない限り操作不能となる
\item けだまは一方向に進み、かべ・あな・アイテム・けだまに接すると反対方向に進む
\end{enumerate}


%%%%%%%%%%%%%%%%%%%%%%%%%%%%%%%%%%%%%%%%%%%%%%%%%%%%%%%%%%
%% subsection 01.05.11 %%%%%%%%%%%%%%%%%%%%%%%%%%%%%%%%%%%
%%%%%%%%%%%%%%%%%%%%%%%%%%%%%%%%%%%%%%%%%%%%%%%%%%%%%%%%%%
\subsection{ようせい}
\begin{enumerate}[label={\sarrow}]
\item 61時限目「もりのようせい」で初登場
\item ようせいはなにもなければ定位置に留まっている
\item ようせいの向いている方向にさとり達がいると、さとり達に向かって突進する
\item さとり達がいなくなると、もとの定位置に向かって進む
\item こいし達がようせいにぶつかると、ライフが1つ減り、ようせいは消滅する
\item さとりの場合はその場で踏まれ、起き上がるまでの間はもとの定位置に向かって進む
\item ようせいはあなの上は通らない
\end{enumerate}


%%%%%%%%%%%%%%%%%%%%%%%%%%%%%%%%%%%%%%%%%%%%%%%%%%%%%%%%%%
%% subsection 01.05.12 %%%%%%%%%%%%%%%%%%%%%%%%%%%%%%%%%%%
%%%%%%%%%%%%%%%%%%%%%%%%%%%%%%%%%%%%%%%%%%%%%%%%%%%%%%%%%%
\subsection{ゆっくりようせい}
\begin{enumerate}[label={\sarrow}]
\item 63時限目「ゆっくりようせい」で初登場
\item ゆっくりようせいはなにもなければその場に留まっている
\item ゆっくりようせいの上下左右の直線上にさとり達がいると、さとり達に向かう
\item 向かっている最中に他の方向にさとり達がいても、そのまま向かい続ける
\item こいし達がゆっくりようせいにぶつかると、ライフが1つ減り、ようせいは消滅する
\item 複数のゆっくりようせいが同時にぶつかった場合、減るライフは1のみ
\item さとりの場合はそのマスで踏まれ続けられ、ゆっくりようせいが消えない限り操作不能となる
\item さとりを踏んでいる間はそのマスから動かない
\item ゆっくりようせいはあなのマスは通れない
\end{enumerate}


%%%%%%%%%%%%%%%%%%%%%%%%%%%%%%%%%%%%%%%%%%%%%%%%%%%%%%%%%%
%% subsection 01.05.13 %%%%%%%%%%%%%%%%%%%%%%%%%%%%%%%%%%%
%%%%%%%%%%%%%%%%%%%%%%%%%%%%%%%%%%%%%%%%%%%%%%%%%%%%%%%%%%
\subsection{ひじり}
\begin{enumerate}[label={\sarrow}]
\item 100時限目「ひじりをたおせ」でのみ登場
\item ひじりはその場で留まって踊り続けている
\item ダメージを3回受けると消滅する
\end{enumerate}


%%%%%%%%%%%%%%%%%%%%%%%%%%%%%%%%%%%%%%%%%%%%%%%%%%%%%%%%%%
%% subsection 01.05.13 %%%%%%%%%%%%%%%%%%%%%%%%%%%%%%%%%%%
%%%%%%%%%%%%%%%%%%%%%%%%%%%%%%%%%%%%%%%%%%%%%%%%%%%%%%%%%%
\subsection{みこ}
\begin{enumerate}[label={\sarrow}]
\item 120時限目「しんのラスボス」でのみ登場
\item ひじりと同様
\end{enumerate}


\clearpage
%%%%%%%%%%%%%%%%%%%%%%%%%%%%%%%%%%%%%%%%%%%%%%%%%%%%%%%%%%
%% section 01.07 %%%%%%%%%%%%%%%%%%%%%%%%%%%%%%%%%%%%%%%%%
%%%%%%%%%%%%%%%%%%%%%%%%%%%%%%%%%%%%%%%%%%%%%%%%%%%%%%%%%%
\modHeadsection{仕掛けの種類}
さとり達を待ち受けるさまざまな仕掛けが存在する。
それらが初めて登場する時限目では、(クリアするまでは)それらを使用したデモムービーが流れる。


%%%%%%%%%%%%%%%%%%%%%%%%%%%%%%%%%%%%%%%%%%%%%%%%%%%%%%%%%%
%% subsection 01.07.01 %%%%%%%%%%%%%%%%%%%%%%%%%%%%%%%%%%%
%%%%%%%%%%%%%%%%%%%%%%%%%%%%%%%%%%%%%%%%%%%%%%%%%%%%%%%%%%
\subsection{あな}
\begin{enumerate}[label={\sarrow}]
\item 5時限目「あなはおちるよ」で初登場
\item さとり達がこのマスに乗ると、落ちる
\item こいし達のいずれかが落ちると、その時点で指導失敗となる
\item さとりが落ちると、画面から消滅し、操作不能となる
\item みず・ようがんのマスもあなと同様
\end{enumerate}


%%%%%%%%%%%%%%%%%%%%%%%%%%%%%%%%%%%%%%%%%%%%%%%%%%%%%%%%%%
%% subsection 01.05.07 %%%%%%%%%%%%%%%%%%%%%%%%%%%%%%%%%%%
%%%%%%%%%%%%%%%%%%%%%%%%%%%%%%%%%%%%%%%%%%%%%%%%%%%%%%%%%%
\subsection{おちるゆか}
\begin{enumerate}[label={\sarrow}]
\item 31時限目「とおるとおちるゆか」で初登場
\item さとり達が通ると、そのマスはあなに変わる
\item あなに変わるまで少しだけラグがあり、それまでは通ることができる
\end{enumerate}


%\clearpage
%%%%%%%%%%%%%%%%%%%%%%%%%%%%%%%%%%%%%%%%%%%%%%%%%%%%%%%%%%
%% subsection 01.05.08 %%%%%%%%%%%%%%%%%%%%%%%%%%%%%%%%%%%
%%%%%%%%%%%%%%%%%%%%%%%%%%%%%%%%%%%%%%%%%%%%%%%%%%%%%%%%%%
\subsection{こおりのゆか}
\begin{enumerate}[label={\sarrow}]
\item 45時限目「すっごいすべるよ」で初登場
\item さとり達が通ると、その先のマスになにもない限り進行方向に強制的に進む
\end{enumerate}


%%%%%%%%%%%%%%%%%%%%%%%%%%%%%%%%%%%%%%%%%%%%%%%%%%%%%%%%%%
%% subsection 01.05.09 %%%%%%%%%%%%%%%%%%%%%%%%%%%%%%%%%%%
%%%%%%%%%%%%%%%%%%%%%%%%%%%%%%%%%%%%%%%%%%%%%%%%%%%%%%%%%%
\subsection[おちるこおりのゆか]{おちるこおりのゆか%
%%% footnote %%%
\footnote{これについてのデモムービーはない。}%
%%%%%%%%%%%%%%%%
}
\begin{enumerate}[label={\sarrow}]
\item 50時限目「すべっておちる」で初登場
\item 「とおるとおちるゆか」と「こおりのゆか」の両方の性質を持つ
\end{enumerate}


%%%%%%%%%%%%%%%%%%%%%%%%%%%%%%%%%%%%%%%%%%%%%%%%%%%%%%%%%%
%% subsection 01.05.10 %%%%%%%%%%%%%%%%%%%%%%%%%%%%%%%%%%%
%%%%%%%%%%%%%%%%%%%%%%%%%%%%%%%%%%%%%%%%%%%%%%%%%%%%%%%%%%
\subsection{スイッチ}
\begin{enumerate}[label={\sarrow}]
\item 53時限目「はじめてのスイッチ」で初登場
\item 赤・青・黄・緑の4種類がある
\item さとり達やアイテムが乗ると、乗っている間だけその色の床が1段上昇する
\end{enumerate}


%\clearpage
%%%%%%%%%%%%%%%%%%%%%%%%%%%%%%%%%%%%%%%%%%%%%%%%%%%%%%%%%%
%% subsection 01.05.14 %%%%%%%%%%%%%%%%%%%%%%%%%%%%%%%%%%%
%%%%%%%%%%%%%%%%%%%%%%%%%%%%%%%%%%%%%%%%%%%%%%%%%%%%%%%%%%
\subsection{ワープするゆか}
\begin{enumerate}[label={\sarrow}]
\item 69時限目「ワープするゆか」で初登場
\item 赤・青・黄・緑の4種類がある
\item さとり達が乗ると、同じ色のワープするゆかにワープする
\item ワープさきにアイテムやさとり達がいると、なにも起きない
\end{enumerate}



\clearpage
%%%%%%%%%%%%%%%%%%%%%%%%%%%%%%%%%%%%%%%%%%%%%%%%%%%%%%%%%%
%% section 01.06 %%%%%%%%%%%%%%%%%%%%%%%%%%%%%%%%%%%%%%%%%
%%%%%%%%%%%%%%%%%%%%%%%%%%%%%%%%%%%%%%%%%%%%%%%%%%%%%%%%%%
\modHeadsection[ライフとダメージ]{ライフ%
%%% footnote %%%
\footnote{「ライフ」についての公式の文言はなく、勝手にそう名付けている。}%
%%%%%%%%%%%%%%%%
とダメージ}


%%%%%%%%%%%%%%%%%%%%%%%%%%%%%%%%%%%%%%%%%%%%%%%%%%%%%%%%%%
%% subsection 01.06.01 %%%%%%%%%%%%%%%%%%%%%%%%%%%%%%%%%%%
%%%%%%%%%%%%%%%%%%%%%%%%%%%%%%%%%%%%%%%%%%%%%%%%%%%%%%%%%%
\subsection{こいし達の場合}

%%%%%%%%%%%%%%%%%%%%%%%%%%%%%%%%%%%%%%%%%%%%%%%%%%%%%%%%%%
%% subsubsection 01.06.01.1 %%%%%%%%%%%%%%%%%%%%%%%%%%%%%%
%%%%%%%%%%%%%%%%%%%%%%%%%%%%%%%%%%%%%%%%%%%%%%%%%%%%%%%%%%
\subsubsection{こいし達のライフ}
\begin{enumerate}[label={\sarrow}]
\item 各々の時限目が始まった段階で、こいし達のライフは3
\item こいし達のいずれかのライフが0になると、消滅し指導失敗となる
\item ダメージを受けると、ライフが1減る
\item たべものを食べると、ライフが全回復(ライフ3)する
\item ライフが1になると、ふらつく
\item 「こいしをつよくしてやりなおす」を選ぶと、こいし達のライフが減らなくなる
\end{enumerate}

%%%%%%%%%%%%%%%%%%%%%%%%%%%%%%%%%%%%%%%%%%%%%%%%%%%%%%%%%%
%% subsubsection 01.06.01.2 %%%%%%%%%%%%%%%%%%%%%%%%%%%%%%
%%%%%%%%%%%%%%%%%%%%%%%%%%%%%%%%%%%%%%%%%%%%%%%%%%%%%%%%%%
\subsubsection{こいし達の受けるダメージ}
以下のような場合に、こいし達はダメージを受ける
\begin{enumerate}[label={\sarrow}]
\item かべにぶつかる
\item けだま・ようせい・ゆっくりようせいにぶつかる
\item こいし達同士でぶつかる
\item 爆風を受ける
\item ひとつうえの段からアイテムが当たる
\end{enumerate}
\begin{marker}
プレゼントに触れている間はダメージを受けない
\end{marker}


%%%%%%%%%%%%%%%%%%%%%%%%%%%%%%%%%%%%%%%%%%%%%%%%%%%%%%%%%%
%% subsection 01.06.04 %%%%%%%%%%%%%%%%%%%%%%%%%%%%%%%%%%%
%%%%%%%%%%%%%%%%%%%%%%%%%%%%%%%%%%%%%%%%%%%%%%%%%%%%%%%%%%
\subsection{けだま・ようせい・ゆっくりようせいの場合}

%%%%%%%%%%%%%%%%%%%%%%%%%%%%%%%%%%%%%%%%%%%%%%%%%%%%%%%%%%
%% subsubsection 01.06.04.1 %%%%%%%%%%%%%%%%%%%%%%%%%%%%%%
%%%%%%%%%%%%%%%%%%%%%%%%%%%%%%%%%%%%%%%%%%%%%%%%%%%%%%%%%%
\subsubsection{けだま・ようせい・ゆっくりようせいのライフ}
\begin{enumerate}[label={\sarrow}]
\item 各々の時限目が始まった段階で、けだま・ようせい・ゆっくりようせいのライフは1
\item ライフが0になると、消滅する
\end{enumerate}

%%%%%%%%%%%%%%%%%%%%%%%%%%%%%%%%%%%%%%%%%%%%%%%%%%%%%%%%%%
%% subsubsection 01.06.04.1 %%%%%%%%%%%%%%%%%%%%%%%%%%%%%%
%%%%%%%%%%%%%%%%%%%%%%%%%%%%%%%%%%%%%%%%%%%%%%%%%%%%%%%%%%
\subsubsection{けだま・ようせい・ゆっくりようせいの受けるダメージ}
以下のような場合に、けだま・ようせい・ゆっくりようせいはダメージを受ける
\begin{enumerate}[label={\sarrow}]
\item こいし達にぶつかる
\item ひとつうえの段からさとり達に踏まれる
\item ひとつうえの段からアイテムが当たる
\end{enumerate}


%%%%%%%%%%%%%%%%%%%%%%%%%%%%%%%%%%%%%%%%%%%%%%%%%%%%%%%%%%
%% subsection 01.06.02 %%%%%%%%%%%%%%%%%%%%%%%%%%%%%%%%%%%
%%%%%%%%%%%%%%%%%%%%%%%%%%%%%%%%%%%%%%%%%%%%%%%%%%%%%%%%%%
\subsection{さとりの場合}
\begin{enumerate}[label={\sarrow}]
\item さとりは基本的にダメージを受けない
\item 爆風を受けた場合、消滅する
\end{enumerate}


\clearpage
%%%%%%%%%%%%%%%%%%%%%%%%%%%%%%%%%%%%%%%%%%%%%%%%%%%%%%%%%%
%% subsection 01.06.03 %%%%%%%%%%%%%%%%%%%%%%%%%%%%%%%%%%%
%%%%%%%%%%%%%%%%%%%%%%%%%%%%%%%%%%%%%%%%%%%%%%%%%%%%%%%%%%
\subsection{ひじり・みこの場合}
\begin{enumerate}[label={\sarrow}]
\item 各々の時限目が始まった段階で、ひじり・みこのライフは3
\item 爆風を受けると、ダメージを受け、ライフが1減る
\item ライフが0になると、消滅する
\end{enumerate}



%\clearpage
%%%%%%%%%%%%%%%%%%%%%%%%%%%%%%%%%%%%%%%%%%%%%%%%%%%%%%%%%%
%% section 01.09 %%%%%%%%%%%%%%%%%%%%%%%%%%%%%%%%%%%%%%%%%
%%%%%%%%%%%%%%%%%%%%%%%%%%%%%%%%%%%%%%%%%%%%%%%%%%%%%%%%%%
\modHeadsection{実績}
以下を満たすと、さとり達のアイコンが表示される
\begin{enumerate}[label={\sarrow}]
\item こいし:100時限目をクリアする
\item おりん:120時限目をクリアする
\item おくう:120時限すべてをクリアする
\item さとり:120時限すべてを優評価でクリアする
\end{enumerate}



%%%%%%%%%%%%%%%%%%%%%%%%%%%%%%%%%%%%%%%%%%%%%%%%%%%%%%%%%%
%% section 01.10 %%%%%%%%%%%%%%%%%%%%%%%%%%%%%%%%%%%%%%%%%
%%%%%%%%%%%%%%%%%%%%%%%%%%%%%%%%%%%%%%%%%%%%%%%%%%%%%%%%%%
\modHeadsection{その他}


%%%%%%%%%%%%%%%%%%%%%%%%%%%%%%%%%%%%%%%%%%%%%%%%%%%%%%%%%%
%% subsection 01.10.01 %%%%%%%%%%%%%%%%%%%%%%%%%%%%%%%%%%%
%%%%%%%%%%%%%%%%%%%%%%%%%%%%%%%%%%%%%%%%%%%%%%%%%%%%%%%%%%
\subsection{さとり達の表情}
盤内のさとり達の状況に応じて、外側のさとり達の表情が変わる

%%%%%%%%%%%%%%%%%%%%%%%%%%%%%%%%%%%%%%%%%%%%%%%%%%%%%%%%%%
%% subsubsection 01.10.02.1 %%%%%%%%%%%%%%%%%%%%%%%%%%%%%%
%%%%%%%%%%%%%%%%%%%%%%%%%%%%%%%%%%%%%%%%%%%%%%%%%%%%%%%%%%
\subsubsection{さとりの表情}
\begin{enumerate}[label={\sarrow}]
\item 踏まれたとき
\item 消滅したとき
\item 目標達成したとき
\end{enumerate}

%%%%%%%%%%%%%%%%%%%%%%%%%%%%%%%%%%%%%%%%%%%%%%%%%%%%%%%%%%
%% subsubsection 01.10.02.1 %%%%%%%%%%%%%%%%%%%%%%%%%%%%%%
%%%%%%%%%%%%%%%%%%%%%%%%%%%%%%%%%%%%%%%%%%%%%%%%%%%%%%%%%%
\subsubsection{こいし達の表情}
\begin{enumerate}[label={\sarrow}]
\item ダメージを受けたとき
\item ライフが1のとき
\item 頭にキノコが生えているとき
\item とうがらしを食べて突進しているとき
\item プレゼントに触れたとき
\end{enumerate}


%%%%%%%%%%%%%%%%%%%%%%%%%%%%%%%%%%%%%%%%%%%%%%%%%%%%%%%%%%
%% subsection 01.10.02 %%%%%%%%%%%%%%%%%%%%%%%%%%%%%%%%%%%
%%%%%%%%%%%%%%%%%%%%%%%%%%%%%%%%%%%%%%%%%%%%%%%%%%%%%%%%%%
\subsection{早送り機能}
ゲーム速度を2倍にする早送り機能がある
\begin{enumerate}[label={\sarrow}]
\item 早送りしている間は、操作できない
\item 多くの時限目で、優評価を得るために必須の機能
\end{enumerate}




%%%%%%%%%%%%%%%%%%%%%%%%%%%%%%%%%%%%%%%%%%%%%%%%%%%%%%%%%%
%% chapters %%%%%%%%%%%%%%%%%%%%%%%%%%%%%%%%%%%%%%%%%%%%%%
%%%%%%%%%%%%%%%%%%%%%%%%%%%%%%%%%%%%%%%%%%%%%%%%%%%%%%%%%%
\modHeadchapter[]{RTAの概要}



%%%%%%%%%%%%%%%%%%%%%%%%%%%%%%%%%%%%%%%%%%%%%%%%%%%%%%%%%%
%% section 02.01 %%%%%%%%%%%%%%%%%%%%%%%%%%%%%%%%%%%%%%%%%
%%%%%%%%%%%%%%%%%%%%%%%%%%%%%%%%%%%%%%%%%%%%%%%%%%%%%%%%%%
\modHeadsection{カテゴリ}
2024/11現在、\href{https://www.speedrun.com/komeiji_satori_no_jousou_kyouiku}{SRC}には以下のカテゴリがある。


%%%%%%%%%%%%%%%%%%%%%%%%%%%%%%%%%%%%%%%%%%%%%%%%%%%%%%%%%%
%% subsection 02.01.01 %%%%%%%%%%%%%%%%%%%%%%%%%%%%%%%%%%%
%%%%%%%%%%%%%%%%%%%%%%%%%%%%%%%%%%%%%%%%%%%%%%%%%%%%%%%%%%
\subsection{any\%}
\begin{enumerate}[label={\sarrow}]
\item 100時限目をクリアする(エンディングに到達する)
\item 言いかえると、こいしのアイコンを取得する
\end{enumerate}


%%%%%%%%%%%%%%%%%%%%%%%%%%%%%%%%%%%%%%%%%%%%%%%%%%%%%%%%%%
%% subsection 02.01.02 %%%%%%%%%%%%%%%%%%%%%%%%%%%%%%%%%%%
%%%%%%%%%%%%%%%%%%%%%%%%%%%%%%%%%%%%%%%%%%%%%%%%%%%%%%%%%%
\subsection{any\% - No Strong Koishi}
\begin{enumerate}[label={\sarrow}]
\item any\%と基本的に同じ
\item ただし、「こいしをつよくしてやりなおす」を使用しない
\item 言いかえると、どの時限目も優または良の評価でクリアする
\end{enumerate}


%%%%%%%%%%%%%%%%%%%%%%%%%%%%%%%%%%%%%%%%%%%%%%%%%%%%%%%%%%
%% subsection 02.01.03 %%%%%%%%%%%%%%%%%%%%%%%%%%%%%%%%%%%
%%%%%%%%%%%%%%%%%%%%%%%%%%%%%%%%%%%%%%%%%%%%%%%%%%%%%%%%%%
\subsection{any\% - Trial}
\begin{enumerate}[label={\sarrow}]
\item 体験版の1-10時限目すべてをクリア
\end{enumerate}
\begin{marker}
体験版は10時限ともすべて本編のものとは異なりオリジナルのものである
\end{marker}



%%%%%%%%%%%%%%%%%%%%%%%%%%%%%%%%%%%%%%%%%%%%%%%%%%%%%%%%%%
%% section 02.02 %%%%%%%%%%%%%%%%%%%%%%%%%%%%%%%%%%%%%%%%%
%%%%%%%%%%%%%%%%%%%%%%%%%%%%%%%%%%%%%%%%%%%%%%%%%%%%%%%%%%
\modHeadsection{SRC以外の記録}
\begin{enumerate}[label={\sarrow}]
\item Syuraさんの\href{https://pastebin.com/0nH3mJSZ}{1-100時限目のPB}
\item 1-120時限目の\href{https://www.nicovideo.jp/watch/sm18710850}{全優評価クリアの動画}(ニコニコ動画)
\end{enumerate}



\clearpage
%%%%%%%%%%%%%%%%%%%%%%%%%%%%%%%%%%%%%%%%%%%%%%%%%%%%%%%%%%
%% section 02.03 %%%%%%%%%%%%%%%%%%%%%%%%%%%%%%%%%%%%%%%%%
%%%%%%%%%%%%%%%%%%%%%%%%%%%%%%%%%%%%%%%%%%%%%%%%%%%%%%%%%%
\modHeadsection{テクニック}
最短手順で進めるのは言うまでもなく、基本的には早送りをどれだけできるかの勝負。
その他には、
\begin{enumerate}[label={\sarrow}]
\item 操作が独特でボタン入力を受け付けないことがあるため、連射機能を使用\\
      慣れるまではミスが頻発するけど、確実にこのほうが早い\\
      連射機能を用いても入力を受け付けないことが稀にある
\item さとりはアイテムをおくよりも1マス移動するほうが早い\\
      最後に踏まれる形でクリアすることが多いのはそのため
\item さとりがアイテムを持ち上げた直後に早送りをすると、0.1秒ほど縮まる(理由は知らない)\\
      アイテムを持ち上げた直後に少しだけ早送りしていることが多いのはそのため\\
      おそらく1-2fくらい早送りになっていればよいかと思われる(体感)
\item ある時限目をクリアし、次の時限目をいかに早く選べるか\\
      よくある失敗が、カーソルが動かせずもとの時限目を選んでしまうミス(秒単位のロス)
\item 時限目の選択で右ボタンを押すと、選択できる最後の時限目にカーソルを移動できる\\
      個人的には右ボタンで選ぶほうがやりやすい\\
      1-50時限目くらいまで選ぶ順番が前後しているのはそのため
\item さとりが踏まれる直前にアイテムをおくと、アイテムをおくモーションが省略され早くなる\\
      より直前であるほど時間の短縮になる\\
      フレーム単位の操作になるのでRTAではまずしない(個別のPB狙いではする)
\end{enumerate}



%\clearpage
%%%%%%%%%%%%%%%%%%%%%%%%%%%%%%%%%%%%%%%%%%%%%%%%%%%%%%%%%%
%% section 02.02 %%%%%%%%%%%%%%%%%%%%%%%%%%%%%%%%%%%%%%%%%
%%%%%%%%%%%%%%%%%%%%%%%%%%%%%%%%%%%%%%%%%%%%%%%%%%%%%%%%%%
\modHeadsection{経緯}
\begin{enumerate}[label={\sarrow}]
\item たまたまお店で見かけ、衝動買い
\item なにこのゲームすてきTAS作りたい
\item でもhourglassで動かない
\item ほなRTAか(フレーム単位の操作なんてそんなないやろ)の精神
\end{enumerate}

\begin{enumerate}[label={\sarrow}]
\item 手順を調べるためにSRCを見ると、ページはあるが記録動画がない(2020年頃)
\item \href{https://www.speedrun.com/users/Syura}{Syuraさん}という方が2015年頃にRTAをしていたらしいが、動画は削除したらしい
\item モデレータが不在で内容も整理されてなく、申請すれば自動的に承認されるような状態に
\end{enumerate}
(数カ月後)
\begin{enumerate}[label={\sarrow}]
\item SASさんが\href{https://www.speedrun.com/komeiji_satori_no_jousou_kyouiku/runs/znol46vy}{RTA}を走られる(56:58)
\item それを受け、参考になればとSyuraさんが5年ほど前の\href{https://www.speedrun.com/komeiji_satori_no_jousou_kyouiku/runs/m3n4v8qy}{記録動画}を再投稿(50:21)\\
      さらに、自身の1-100時限目のPBを\href{https://pastebin.com/0nH3mJSZ}{公開}
\item それを受け、SASさんが自身の記録を\href{https://www.speedrun.com/komeiji_satori_no_jousou_kyouiku/runs/m3ro3edm}{更新}(49:57)
\item これら4つの記録と、ニコニコ動画にあった\href{https://www.nicovideo.jp/watch/sm18710850}{全優評価クリアの記録}をexcelにまとめて比較
\end{enumerate}


%%%%%%%%%%%%%%%%%%%%%%%%%%%%%%%%%%%%%%%%%%%%%%%%%%%%%%%%%
%% Appendices %%%%%%%%%%%%%%%%%%%%%%%%%%%%%%%%%%%%%%%%%%%
%%%%%%%%%%%%%%%%%%%%%%%%%%%%%%%%%%%%%%%%%%%%%%%%%%%%%%%%%
\begin{appendices}
%\Appendixpart
\end{appendices}

\addtocontents{toc}{\protect\end{tocBox}}
