%!TEX root = ../KSnJK.tex


%%%%% FOR STYLE OF PART %%%%%%%%%%%%%%%%%%%%%%
\renewcommand\partpagestyle{emptydate}

%%%%% TIKZSET %%%%%%%%%%%%%%%%%%%
\tikzset{
  %%%%% FORMAT STYLE %%%%%
  commonComponents/.style={signal, draw, signal to=nowhere},
  chapter/.style={
    fill=\if@appendix white\else Maroon!65\fi,
    text=\if@appendix Maroon!65\else white\fi,
    signal from=east,
    inner sep=5pt,
    text height=1.5ex,
    text depth=2pt
  },
  sect/.style={text depth=0pt, text=white},
  section/.style={sect, fill=konpeki!100!, signal to=east, inner sep=5pt},
  subsection/.style={sect, fill=moegi!90!, signal to=nowhere, inner sep=5pt},
  subsubsection/.style={sect, fill=sssec!100!, signal to=nowhere, inner sep=4pt},
}

%%%%% FOR STYLE OF CHAPTER %%%%%
\renewcommand\chapterpagestyle{\if@frontmatter plainheadfront\else plainhead\fi}
\renewcommand*{\chapterformat}{%
  \begin{tikzpicture}[every node/.style={commonComponents}, baseline=-7.75pt] % +: title up, -: title down
    \node[chapter] at (0, 0) {\LARGE{\sffamily\bfseries\,\thepart.\,\thechapter}};
  \end{tikzpicture}~~%
}

%%%%% FOR STYLE OF SECTION %%%%%%%%%%%%%%%%%%%
\setcounter{secnumdepth}{4}
\renewcommand*{\sectionformat}{%
  \begin{tikzpicture}[every node/.style={commonComponents}, baseline=-4.75pt]
    \node[section] at (0, 0) {\Large{\sffamily\bfseries\,\thesection~\relax}};
  \end{tikzpicture}~~%
}

%%%%% FOR STYLE OF SUBSECTION %%%%%%%%%%%%%%%%
\renewcommand*{\subsectionformat}{%
  \begin{tikzpicture}[every node/.style={subsection}, baseline=-4.25pt]
    \node at (0, 0) {\large{\sffamily\bfseries\thesubsection}};
  \end{tikzpicture}~~%
}

%%%%% FOR STYLE OF SUBSUBSECTION %%%%%%%%%%%%%
\renewcommand*{\subsubsectionformat}{%
  \begin{tikzpicture}[every node/.style={subsubsection}, baseline=-2.5pt]
    \node at (0, 0) {\sffamily\bfseries\thesubsubsection};
  \end{tikzpicture}~~%
}

%%%%% FOR STYLE OF PARAGRAPH %%%%%%%%%%%%%%%%%
%for scrbook.cls
\RedeclareSectionCommand[%
  style=section,%
  level=4,%
  indent=0pt,%
  afterindent=false,
  beforeskip=3.25ex \@plus1ex \@minus.2ex,%
  afterskip=0.1ex \@plus.1ex \@minus.1ex,% -1em から変更
  tocindentfollows=subsubsection,%
  tocstyle=section,%
  tocindent=10em,%
  tocnumwidth=5em,% def: 5em
  font=\raggedsection\textgt\nobreak\sball{blue}~
]{paragraph}
%for book.cls
%\renewcommand\paragraph[1]{%
%  \@startsection{paragraph}{\paragraphnumdepth}{0pt}%
%  {3.25ex \@plus1ex \@minus.2ex}% \@plus, \@minusは伸び縮みできるスペースの長さ
%  {0.1ex\@plus.1ex \@minus.1ex}% ここが正だと改行されて、値だけ垂直スペースが入る
%  {\raggedsection\normalfont\sectfont\gtfamily\nobreak\size@paragraph\sball{blue}~}{#1}\noindent
%}

%%%%% FOR STYLE OF SUBPARAGRAPH %%%%%%%%%%%%%%
\RedeclareSectionCommand[%
  style=section,%
  level=5,%
  indent=0pt, % default: \scr@parindent,%
  afterindent=false,
  beforeskip=0.5ex \@plus1ex \@minus.2ex,% 3.25ex \@plus1ex \@minus .2ex から変更
  afterskip=0.1ex \@plus.1ex \@minus.1ex,% -1em から変更
  tocstyle=section,%
  tocindent=12em,%
  tocnumwidth=6em%
]{subparagraph}
